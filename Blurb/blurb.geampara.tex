% TODO
%  - add ':' in descriptions; config as default
%  - add chord names to arpeggiation examples
% TODO
%   - for items, make no separation the default
%   - for description items, add a ':'
%   - remove numbers / maybe tweak font on subsections
\documentclass[landscape, 12pt]{article}

\usepackage[landscape,margin=1.5cm]{geometry}
\usepackage{lyluatex}
\usepackage{multicol}
\usepackage{enumitem}
\usepackage{titlesec}

\titleformat{\section}{\normalfont\Large\bfseries}{}{0pt}{}

\setlength{\parindent}{0pt}
\setlength{\columnseprule}{.4pt}

\thispagestyle{empty}

\begin{document}
\raggedcolumns % this isn't working as expected
\begin{multicols}{3}



\section{Rhythm}
The rhythm of these tunes is parsed either as\
\begin{center}
\lilypond{\time 7/16 d'16[ d' d' d'] d'[ d' d']}
\end{center}
or
\begin{center}
\lilypond{\time 7/16 d'16[ d'] d'[ d'] d'[ d' d']}
\end{center}

\section{Basslines from Chords}

\begin{description}

\item[major]
A way to play both major and minor chords:
\begin{center}
\lilypond{\key d \major \clef bass \time 7/16 d8[ d'] a[ d'16]}
\end{center}

\item[$7^{th}$s as passing notes] \ 
\begin{center}
\lilypond{\clef bass \time 7/16 a,8[ a] c[ e16]}
\end{center}
or
\begin{center}
\lilypond{\clef bass \time 7/16 a,8[ a] c[ a16]}
\end{center}

\item[opposite of $7^{th}$s as passing] \ 
\begin{center}
\lilypond{\clef bass \time 7/16 a,8[ a] b[ d16]}
\end{center}
\end{description}

\section{Ornaments}
TODO

\section{Notes on Tunes}
\begin{description}[noitemsep]
\item[Bulcenska vs Bulčenska R.:]
	Two distinct tunes.
\item[Bârla, G.\ din:]
	Highly chromatic.
\item[Comida La Mañana, La:]
	Ladino.
\item[Culesul Viilor, La:]
	Play like musette.
	Highly chromatic.
\item[Čekurjankino Horo:]
	Speeds up as it goes.
\item[Dobrogeana:]
	\dots
\item[Haidim, G.\ lui:]
	Unlike all the other tunes that are `deranged',
	this one keeps the vamping and alternations since
	they're so good.
\item[Hijaz:]
	Oddly not in the hijaz scale.
\item[Mala Loka:]
	\dots
\item[Murfatlar, G.\ de la:]
	2 distinct tunes with
	the name differing (G.-ua vs G.-lele)
\item[Ostropesul:]
	Play like a renaissance song
\item[Taşaul, G.\ de la:]
	Same as G.\ de la Babadag
\item[Turcitu, G.\ de la:]
	Has some similarity to the nameless G.\ from BGKO.
\item[Ţigănică:]
	\dots
\end{description}

\section{Todo}
\begin{itemize}
\item add G with gis a a e gis a e
\item
\item
\item
\item
\item
\item
\item
\item
\item
\item
\item
\item
\item
\item
\item
\item
\item
\item
\item
\item
\item
\item
\item
\end{itemize}

\include{Blurb/footer.tex}
% geampara: Fast tempo Romanian aksak 223.
% mandilatos: Fast Greek Thracian aksak 223.
% ruchenitsa: Moderate to fast tempo Bulgarian aksak 223.
% serranitsa: Fast tempo Pontic aksak 223.
